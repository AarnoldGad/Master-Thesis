\chapter*{Acknowledgements}
\addcontentsline{toc}{chapter}{Acknowledgements}

\renewcommand{\epigraphflush}{flushright}
\setlength{\epigraphwidth}{0.6\textwidth}
\myepigraph{Help will always be given [...] to those who ask for it.}{J.K. Rowling,\\\textit{Harry Potter and the Chamber of Secrets}}
\vspace{2\baselineskip}

I wish to thank my supervisor, Evan O'Connor, for giving me the opportunity to work on this project, and supervising it. I thank him for his help, his feedback during the writing of this thesis and for taking the time to answer all my questions. I also wish to thank Anders Jerkstrand for making sure that I found a project that I liked. I wish to thank Haakon Andresen, whose advice and tremendous support greatly helped me during this project, as well as outside of it. I thank Oliver Eggenberger Andersen too, for his help and for providing me with some of his Jupyter Notebooks as examples. Also from the astronomy department, I wish to thank Nariman Nik Khah for the interesting conversations at lunch, and Maria Youngman for always smiling and looking happy (that really puts you in a good mood in the morning).

Of course, I thank all my fellow Master students with which I had an amazing experience in Stockholm. In particular Keyur Vithlani, who helped me during the whole Master and even provided me with a few good references for this thesis, and Gustaf Beckman Berg for being Gustaf Beckman Berg. I also thank Alice Knutas for making circles.

Lastly, I wish to thank my family for supporting me in my studies in countless different ways. I thank my mother, Christelle, for taking the decision to home-school me, and my father, Jean-Max, for trusting me and letting me fiddle with my computer instead of doing my homework. And finally, I thank my sister, Maeva, who always pushed me forward and inspired me. I would simply not have made it here without her.

The computations were performed at NSC Tetralith provided by the National Academic Infrastructure for Supercomputing in Sweden (NAISS), partially funded by the Swedish Research Council through grant agreement no. 2022-06725.
