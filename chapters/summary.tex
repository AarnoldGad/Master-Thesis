\mainchapter{Epilogue}{0.6}{I suffered often, I was wrong sometimes, but I loved.\\It is I who have lived and not an illusion created by my pride and my boredom.}{Alfred de Musset, \\\textit{On ne badine pas avec l'amour}} \label{chap:outro}
%{0.35}{Points to Ponder: ...}{Evan O'Connor}

%----------------------------------------%
% Epilogue
%----------------------------------------%

Performing long-term simulations of core-collapse supernovae self-consistently from first principles is too computationally expensive. A common approach to this problem is to use an inner boundary condition to relax the time step constraint imposed on the simulation by the central cells, near and inside the proto-neutron star. However, little work has been done in this direction and often consist of coarse approximations that do not model well the conditions at the inner boundary.

We have implemented into the state-of-the-art \flash\ simulation code a new inner boundary condition modelling an isotropic neutrino-driven wind forming near the proto-neutron star, following the description of \cite{Wongwathanarat2015} and \cite{Stockinger2020}. We have performed 1D and 2D simulations until shock breakout and analysed the impact of the boundary condition and its parametrisation on the evolution. We have found that the neutrino wind has a non-negligible influence on the inner ejecta and the evolution of the diagnostic explosion energy during the first few seconds after bounce, corresponding to the cooling of the PNS. However, the mapping of the inner boundary can significantly interfere with the hydrodynamics in the presence of strong, asymmetrical downflows and outflows in the central region. In particular, the results from our 2D simulations indicate that our time of mapping, at \(1\units{s}\) after bounce, was too early, as strong and aspherical accretion channels are still active at that time and complicate the introduction of our spherical neutrino wind boundary condition. Mapping at later times, when accretion has weakened and the explosion energy reaches an asymptotic value, should solve this problem. It is however uncertain how long this can take, and may represent a non-negligible computational cost. In addition, the dependence on the progenitor is also unclear, and low-mass progenitors could more easily develop a spherical wind while more massive progenitors may experience stronger and longer-lasting accretion. Consequently, we find that the assumption of a spherically symmetric neutrino wind is limiting. Moreover, \cite{Wang2023} showed the formation of aspherical neutrino winds in their 3D simulations, leading to significant injections of energy into the explosion. This challenges the isotropic wind approximation, and calls for a more complex, multidimensional modelling of our inner boundary condition.

Apart from the neutrino-driven wind, alternative parametrisation of the inner boundary can be used to explore more exotic central engines, such as magnetars and disk-driven winds forming from an accreting black hole.
