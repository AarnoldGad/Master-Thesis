\mainchapter{Discussion}{0.6}{Your calculations are correct, but your physical insight is \flqq{} Tout à fait abominable \frqq{}.}{Albert Einstein to Georges Lemaître} \label{chap:discussion}

%----------------------------------------%
% Discussion
%----------------------------------------%

\section{Necessity of the boundary condition}

Current state-of-the-art simulation codes for CCSNe, such as \flash, are now capable of producing successful explosions with the neutrino-driven mechanism in multiple dimensions. This mechanism requires contributions from various areas of physics, including notably (magneto-)hydrodynamics, nuclear reactions, neutrino transport, general relativity, and a treatment of the equation of state of matter in both nuclear- and low-density regimes. As a result, such simulations come at a large computational cost which increases exponentially with the number of dimensions. For this reason, if 2D simulations using axisymmetry are now more common, full 3D simulations are still quite rare. Moreover, these simulations rarely extend beyond hundreds of milliseconds to a few seconds after bounce, and attention is usually given to the explosion mechanism. This computational limitation is mainly a consequence of the high resolution requirement in the central region containing the neutron star, that imposes heavy constraints on the time step required to evolve the simulation. Excising the central region from the domain is a simple yet effective solution to extend a simulation to late times at a reduced computational cost, and is the approach that we have adopted in this thesis. However, the question persists as to what should replace this region, in other words, what condition on the evolution should be put at the boundary. For example, \cite{Sandoval2021} performed simulations until late times using a diode boundary condition, similar to our description of the inflow boundary condition in \Cref{sec:bdry_in}, that exclusively allows matter to flow out of the domain. We note that in their paper, \cite{Sandoval2021} define \emph{inflow} and \emph{outflow} as matter flowing in or out of the domain, in opposition with the definition we use in this thesis, as mentioned in \Cref{sec:lt_setup}. They showed that this simple parametrisation can give valuable insights on the evolution of the initial ejecta, in particular the formation of metal-rich bullets from Rayleigh-Taylor instabilities at the shock front, and on the final morphology of the explosion. In contrast, \cite{Stockinger2020} and \cite{Wongwathanarat2015} have performed similar simulations using an outflow boundary condition mimicking the behaviour of a spherical neutrino-driven wind around the central engine, and is the description that we have based our inner boundary condition on, as described in \Cref{sec:bdry_out}. We have further explored the use of this boundary condition and analysed its effects on the simulation by modifying parameters such as the wind velocity. Also, our boundary condition exhibits an interesting property in that it evolves the cells within the inner region, thus allowing accretion onto the boundary.

\section{Validity \& Self-consistency}

Clearly, we cannot achieve self-consistency with an inner boundary condition, and the resulting simulation is bound to be an approximation. Parts of the explosion, such as the forward shock, are less affected by the introduction of an inner boundary condition, and can still provide relevant observable quantities on the late time evolution of the supernova. On the contrary, the impact on the more central regions, up to the reverse shock, and on the estimation of the nucleosynthetic yields from the inner ejecta can become non-negligible. In order to achieve long-term simulations that are as accurate as possible, a simple but realistic parametrisation of the inner boundary is required.

An inflow condition is perhaps the most simple parametrisation, but is not very suited unless the central engine is a black hole, as it causes most of the inner ejecta behind the reverse shock to fall back and accrete onto the central region. In the presence of a neutron star central engine, we require a more appropriate parametrisation capable of supporting this material from below, making a neutrino-driven wind a more reasonable choice. Furthermore, the neutrino-driven wind is expected to form due to the emission of neutrinos by the PNS, and this behaviour has been observed in several recent multidimensional simulations of CCSNe \citep{Witt2021,Navo2023,Wang2023}. \cite{Wang2023} show that the wind is a common feature of long-term CCSN simulation and demonstrate their existence. Moreover, they show that this wind can carry 10\%-20\% of the explosion energy.

\section{Main results \& limitations}

\subsection{Dimensionality}

The neutrino-driven wind is usually defined as an isotropic, spherically symmetric neutrino-heated outflow originating from the surface of the PNS. This is a standard feature of exploding 1D simulations, due to the inherent spherical symmetry of the problem, but is less well defined in multiple dimensions. As we have seen in \Cref{sec:results_2d}, our 2D simulations show very aspherical combinations of downflows and outflows in the central region surrounding the PNS before mapping. Channels of accretion can be beneficial as they boost the neutrino luminosities which, in turn, can power strong outflows in the directions that are not obstructed by infalling material. We have seen that this was particularly true in the case of our initial s12hf1p0 simulation, but that the s12hf1p5 and s12hf2p0 simulations more easily drive material out before mapping due to the higher neutrino heating efficiencies. In their paper, \cite{Wang2023} emphasise that the wind is not required to be spherical, but can have various morphologies shaped by the interaction with the surrounding material, from a concentrated, cone-like region to more spread-out channels of ejection. Our spherical neutrino wind approximation is therefore very limited and should be reconsidered in multiple dimensions to better fit the conditions from the initial simulation.

\subsection{Wind velocity}

We have found that the aspherical wind forming in the 2D simulations before mapping complicates the choice of wind velocity at the inner boundary. The simulations show that typical values of the wind velocity range from \(10{,}000\units{km/s}\) to \(40{,}000\units{km/s}\), thus we have run multiple simulations with velocities in this range. We have seen that this choice influences multiple aspects of the simulation.

\begin{description}
    \item[Diagnostic explosion energy] The wind velocity influences significantly the evolution of the diagnostic explosion energy during the first few seconds after mapping. A wind too weak, such as in our s12hf1p0\_w10e8 simulation, results in an immediate drop in explosion energy, analogous to that observed in the case where the PNS collapses and a black hole forms, whereas a much stronger wind, such as in the cases of the s12hf1p0\_w28e8 and s12hf1p5\_w15e8 simulations, better reproduces the continued rise in explosion energy, as observed in the initial simulation for a few \(100\units{ms}\) after mapping, and reaches an asymptotic value within \(\sim8\units{s}\) after bounce. However, we have seen that s12hf2p0 was able to keep a nearly constant explosion energy with a relatively weak wind. This is because the higher heating factor rendered the central region more prone to hosting our boundary condition at time of mapping, with already very little accretion and a neutrino wind forming in the south hemisphere as indicated by the presence of a termination shock.

    \item[Accretion] In the same way, and as can be expected, we have seen that stronger winds limit the amount of accretion onto the inner boundary. However, the wind weakens rapidly within the first \(\sim10\units{s}\) as the density prescribed at the inner boundary drops, becoming less capable of accelerating material outwards and counteracting accretion.

    \item[Hydrodynamics] Lastly, we have seen in the case of our s12hf1p0\_w28e8 simulation that the strong wind velocity and the very aspherical conditions at mapping cause a shock to be launched from the inner boundary, interfering with the hydrodynamical evolution of the simulation. This shock propagated at supersonic speed and reached the forward shock, perturbing the explosion. However, as we have shown in \Cref{sec:breakout}, the explosion energy can be linked to the shock breakout time, therefore the neutrino-driven wind inevitably leads to a change in the hydrodynamics as it injects more or less energy into the explosion, affecting the shock velocity. This is however not unwanted as the objective is to match as closely as possible the expected behaviour of the simulation if it was simulated self-consistently, without a boundary condition.
\end{description}

\section{Possible improvements}

Possibly the most important of these results is the impact on the hydrodynamics. This directly sets an upper limit on the wind velocity that we can prescribe if it causes the formation of a shock that interferes significantly with the shape of the forward shock. A prescription capable of preventing excess accretion while having little impact on the overall hydrodynamics and, consequently, the explosion energy, such as is the case in our s12hf2p0 simulation, is more desirable. In the case of s12hf2p0, however, most of the accretion was already shut off at moment of mapping due to the enhanced neutrino heating efficiency. A possible solution would therefore be to map at a later time of the simulation, such as \(2\text{-}3\units{s}\) after bounce, when accretion naturally slows down. However, this involves an additional, potentially non-negligible computational cost, which would be particularly limiting in the case of 3D simulations. There is no guarantee that the conditions at the inner boundary would become more favourable later. For example, \cite{Witt2021} show that most of their 2D simulations do not develop steady neutrino winds and that significant downflows can last for over \(3\units{s}\) after bounce. The simulations that do form a neutrino wind only do so after \(3\text{-}6\units{s}\) post-bounce.

We have also discussed that the neutrino wind can adopt various shapes in the case of multidimensional simulations. Our spherically symmetric neutrino wind is perhaps too simplistic, and a more complex, multidimensional boundary condition could be more appropriate. This multidimensional boundary condition could then integrate regions of accretion and ejection of material that would better match the conditions at time of mapping. The boundary could then, for example, smoothly shut off the accretion and restore its spherically symmetric character using a time dependent prescription on the velocity. Such a boundary condition needs to be carefully modelled, but has more potential at adapting to highly progenitor dependent conditions. It is not clear however what the dependence of the neutrino wind on the progenitor star is. \cite{Bollig2021} did not observe the development of a spherical neutrino wind in their 3D simulation of a \(19\sunmass\) progenitor even after \(7\units{s}\) post-bounce. They argue that an isotropic wind could principally be a feature of progenitors in the low-mass end of the CCSN range with low core compactness, as these progenitors only produce low-energy explosions of \(0.1\text{-}0.2\units{B}\). Conversely, long-lasting accretion channels powering neutrino-heated outflows are essential to producing powerful explosions in more massive stars. Furthermore, we have performed 1D and 2D simulations, but the 3D case must be considered as well. \cite{Muller2015} compared the dynamics of the material surrounding the PNS after shock revival in 2D and 3D. They showed that 2D simulations are more prone to forming laminar flows, where the interfaces between the accretion channels and the neutrino-heated material stays stable for a longer period of time, such as can be observed in \Cref{fig:s12hf1p0_og}. Conversely, 3D simulations of the same progenitor will be affected by the Kelvin-Helmholtz instability and rapidly disrupts these interfaces, leading to more turbulent behaviours of the material. As a result, accretion may be more easily shut off in 3D simulation, facilitating the introduction of our boundary condition even at early times.

\section{Outlook}

In this thesis we have presented and discussed an inner boundary condition based on the development of a neutrino-driven wind around the PNS, but other parametrisation are possible. Moreover, certain parameters, such as rotation and magnetic fields, have not been explored at all in this work. With the final aim of linking the simulated explosion mechanism and progenitor properties to observations, performing long-term simulations with a more specific parametrisation can provide valuable insights on certain exotic supernovae and transient events. For instance, the origin of Gamma-Ray Bursts (GRBs) is not very well constrained, but most likely associated with the death of massive stars. A possible scenario is the \emph{collapsar} scenario, where a massive star undergoes a failed supernova and forms a black hole. The infalling material then forms an accretion disk around the black hole and can provide a substantial amount of energy to the supernova as well as power relativistic jets causing a GRB. Such scenario was explored by \cite{Menegazzi2024} on a \(20\sunmass\), rotating progenitor star using an inner boundary condition modelled after the disk wind, where an outflow driven by the accretion disk is forced on an angle around the equator, while material is only allowed to infall towards the central engine at the poles.

Another possible scenario associated with GRBs is the \emph{magnetar} scenario. Magnetars are highly magnetised neutron stars with magnetic fields of the order of \(10^{14}\text{-}10^{15}\units{G}\), and could represent \(\sim10\%\) of newly born neutron stars \citep{Kouveliotou1998}. It is argued that fast-spinning neutron stars, with periods of a few milliseconds, could achieve magnetar fields. The two properties of fast rotation and strong magnetic fields would naturally lead to strong relativistic jets powering GRBs. Hypernovae, very powerful supernovae with explosion energies near \(\sim10\units{B}\), could be the birth place of these fast rotators, which in turn would be powered by tapping in the spin kinetic energy of the proto-magnetar \citep{Burrows2021}. It was shown that magnetar central engines can deposit a significant amount of energy into the supernova and substantially brighten the light curve \citep{Kasen2010}. An inner boundary condition modelled after the spin-down of a magnetar central engine could provide valuable information on such events as GRBs, hypernovae and Super-Luminous Supernovae (SLSNe).
