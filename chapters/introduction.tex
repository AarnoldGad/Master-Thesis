\mainchapter{Introduction}{0.65}{The supernova problem has been solved many times in the last thirty years, but never yet for long.}{Burrows, Hayes \& Fryxell, \\ \textit{On the Nature of Core-Collapse Supernova Explosions}} \label{chap:intro}

%----------------------------------------%
% Introduction
%----------------------------------------%

Supernovae are amongst the brightest and most energetic events in the Universe, and can be as luminous as an entire galaxy. Certain Galactic supernovae such as SN 1054, whose remnant is still observable today as the Crab nebula, took place close enough from the Earth to be observed in plain day with the naked eye \citep{Hamacher2014}. In this thesis, we focus on a specific category of supernovae known as Core-Collapse Supernovae (CCSNe). This type of supernovae arises from the gravitational collapse of the core of dying massive stars into neutron stars, which releases tremendous amounts of gravitational energy in the form of kinetic energy and neutrinos, exploding the entire star and leaving only the neutron star remnant.

These events are devastating for the star and its surroundings, but are nonetheless an essential part of galactic evolution and constitute an interest in various fields of physics and astronomy. For example, most of the heavy elements up to the iron peak are formed through nuclear fusion processes taking place in the core of massive stars. The explosive end of these stars permits these elements to be released in the interstellar medium and chemically enrich the gas from which new stars will form. Heavy elements, beyond the iron peak, are also likely to form during the supernova explosion itself through processes such as explosive nucleosynthesis and rapid neutron capture \citep{Arcones2023}; The expanding shock front in supernova remnants can accelerate particles to relativistic speeds through a process known as Fermi acceleration, and evidence was found that they are an important source of Galactic cosmic rays \citep{Ackermann2013}; CCSNe are the birth places of neutron stars and stellar black holes, whose mergers are essential to gravitational wave astronomy, and were shown to be a site of rapid neutron capture nucleosynthesis \citep{Abbott2017}; Lastly, the extreme conditions in neutron stars makes them and their formation during a supernova interesting laboratories to explore the properties and the state of matter at extremely high densities and temperatures. The study of supernovae is crucial, but is hindered by a few major obstacles: the inability to predict when or where a star will explode as a supernova, and the impossibility to observe the details of the explosion mechanism taking place in the core using conventional telescopes. As a result, we often need to resort to simulations if we wish to better understand the complexity of supernovae and challenge theories. However, accurate simulations of supernovae require the use of multiple physics, such as hydrodynamics, neutrino transport, nuclear processes and a treatment of general relativistic effects, and performing these simulations in multiple dimensions come with a large computational cost. As a result, most studies only focus on the first few seconds of the explosion, that can be self-consistently simulated in a reasonable time, and long-term simulations are rare. Nevertheless, the ability to pursue these simulations until late times to follow the expansion of the supernova shock wave in the outer layers of the star and through its breakout from the stellar surface is crucial to link simulations to observations.

In this thesis, we modified a state-of-the-art simulation code for CCSNe and implemented a new inner boundary condition that we used to perform long-term simulations until shock breakout at a reduced computational cost. The inner boundary condition is modelled to mimic an isotropic neutrino-driven wind emanating from the proto-neutron star, which we based on the description by \cite{Wongwathanarat2015} and \cite{Stockinger2020}. We tested the implementation and analysed the impact of the approximation on the subsequent evolution of the supernova explosion.

We begin in \Cref{chap:theory} with a review of the essential theories, including the basics of astrophysical fluid dynamics, stellar evolution and supernovae. We also present the evolution and details of numerical simulations of CCSNe, and discuss the theory of neutrino-driven wind. In \Cref{chap:methods}, we present \flash, the simulation code that we used, and the details of the simulations. The inner boundary condition and its parametrisation for each simulation is developed as well as other numerical details on the implementation. Then, we show the results of our simulations in \Cref{chap:results}, and discuss these results and the limitations of our implementation in \Cref{chap:discussion}. We then propose possible improvements and explore other eventual uses of our boundary condition.
